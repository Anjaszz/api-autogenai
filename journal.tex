```
\documentclass{article}

\title{Kendaraan Listrik: Masa Depan Transportasi Ramah Lingkungan}
\author{Anjaszzz}
\begin{document}

\maketitle

\raggedright 
\section*{Abstrak}
Kendaraan listrik (EV) menjadi semakin populer sebagai alternatif ramah lingkungan untuk kendaraan berbahan bakar bensin dan diesel. EV tidak menghasilkan emisi gas rumah kaca, sehingga mengurangi polusi udara dan perubahan iklim. Artikel ini membahas perkembangan teknologi EV, manfaat lingkungannya, dan tantangan yang dihadapi dalam adopsi EV secara luas.

\newline
\newline
\textbf{Kata Kunci} : Kendaraan Listrik, Emisi Nol, Transportasi Berkelanjutan

\section{Pendahuluan}
Sektor transportasi merupakan penyumbang emisi gas rumah kaca yang signifikan, yang berkontribusi terhadap perubahan iklim. Kendaraan berbahan bakar bensin dan diesel melepaskan karbon dioksida (CO2), nitrogen oksida (NOx), dan partikulat ke atmosfer, yang berdampak negatif pada kesehatan manusia dan lingkungan. Kendaraan listrik menawarkan solusi untuk mengurangi emisi ini dan mempromosikan transportasi yang lebih berkelanjutan.

\section{Teknologi Kendaraan Listrik}
EV menggunakan motor listrik yang ditenagai oleh baterai isi ulang. Baterai EV menyimpan energi listrik yang digunakan untuk menggerakkan motor dan memberi tenaga pada fitur kendaraan lainnya. EV tidak memiliki sistem pembuangan, sehingga tidak menghasilkan emisi gas buang.

\section{Manfaat Lingkungan Kendaraan Listrik}
Manfaat lingkungan utama dari EV adalah pengurangan emisi gas rumah kaca. Dengan tidak menghasilkan emisi CO2, EV membantu mengurangi polusi udara dan perubahan iklim. Selain itu, EV tidak mengeluarkan NOx atau partikulat, sehingga meningkatkan kualitas udara dan kesehatan masyarakat.

\section{Tantangan Adopsi Kendaraan Listrik}
Meskipun memiliki manfaat lingkungan yang signifikan, EV menghadapi beberapa tantangan dalam adopsi secara luas. Salah satu tantangan utama adalah biaya awal yang lebih tinggi daripada kendaraan berbahan bakar bensin atau diesel. Tantangan lainnya termasuk keterbatasan jangkauan, waktu pengisian daya yang lama, dan kurangnya infrastruktur pengisian daya.

\section{Kesimpulan}
Kendaraan listrik menawarkan potensi besar untuk mengurangi emisi gas rumah kaca dan mempromosikan transportasi yang lebih berkelanjutan. Namun, masih ada tantangan yang perlu diatasi untuk mendorong adopsi EV secara luas. Kolaborasi antara pemerintah, industri, dan konsumen sangat penting untuk mengatasi tantangan ini dan mewujudkan masa depan transportasi listrik yang bersih dan ramah lingkungan.

\section{Daftar Pustaka}
\begin{enumerate}
    \item Birol, F. (2019). Global EV Outlook 2019: Scaling Up Electric Vehicle Deployment. International Energy Agency.
    \item Hawkins, T. R., Singh, B., Majeau-Bettez, G., & Str�mman, A. H. (2012). Comparative Environmental Life Cycle Assessment of Conventional and Electric Vehicles. Journal of Industrial Ecology, 16(1), 337-350.
    \item IEA. (2020). Global Electric Vehicle Outlook 2020. International Energy Agency.
\end{enumerate}

\end{document}
```